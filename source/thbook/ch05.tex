\eject
\vbox{\pic[93.2mm]{thom.jpg}}
\vskip-3cm %-2.35cm
{\tenitbx\baselineskip15pt\leftskip2mm\rightskip6.8cm\parindent4.6cm
Jesus said, `Let him who seeks continue seeking until he
finds. When he finds, he will become troubled. When he becomes
troubled, he will be astonished, and he will rule over the all.'
\hfill\eightrm ---Gospel according to Thomas, {\eightmit 2}nd century\par
}

\chapter Changing layout of PDF maps.

This chapter is extremely useful if you're not satisfied with the predefined 
layout of map symbols and maps provided, and want to adapt them to your needs.
However, you need to know how to write plain \TeX\ and \MP\ macros to do this.


\subchapter Page layout in the atlas mode.

The |layout| command allows basic page setup in the atlas mode. This is done 
through its options such as |page-setup| or |overlap|. But there are no 
options which would specify the position of map, navigator and other elements 
inside the area defined by |page-width| and |page-height| dimensions; e.g., 
why is the navigator below the map and not on its right or left side?

There are many possible arrangements for a page. Rather than offer even more
options for the |layout| command, Therion uses the \TeX\ language to 
describe other page layouts. 
%(See the section `layout' where to put \TeX\ commands.) 
This approach has the advantage that the user has 
direct access to the advanced typesetting engine without making the language of 
Therion overcomplex.

Therion uses pdf\TeX\ with the {\it plain} format for typesetting. So you 
should be familiar with the plain \TeX\ if you wish to define new layouts.

The ultimate reference for plain \TeX\ is
\list
  Knuth, D. E.: {\it The \TeX book}, 
                Reading, Massachusetts, Addison-Wesley $^1$1984 %[...]
\endlist

For pdf\TeX's extensions there is a short manual
\list
  Th\`anh, H. T.---Rahtz, S.---Hagen, H.: {\it The pdf\TeX\ user manual}, 
                available at  \hfil\break 
		\www{http://www.pdftex.org}
\endlist

The \TeX\ macros are used inside of |code tex-atlas| part of the |layout| 
command (see the chapter 
{\it Processing data} for details). The basic one predefined by Therion is the

|\dopage|

macro. The idea is simple: for each page Therion defines \TeX\ variables 
(count, token, and box registers) which contain the page elements (map, 
navigator, page name etc.). At the end of each page macro |\dopage| is invoked. 
This defines the position of each element on the page. By redefining this macro 
you'll get desired page layout. Without this redefinition you'll get the 
standard layout.

Here is the list of variables defined for each page:

\list
  {\it Boxes:}

  * |\mapbox| = The box containing the map. Its width (height) is set 
    according to the |size| and |overlap| options of the |layout| command to 

    |size_width + 2*overlap| or 
    
    |size_height + 2*overlap|, respectively

  * |\navbox| = The box containing the navigator, with dimensions

    |size_width * (2*nav_size_x+1) / nav_factor| or

    |size_height * (2*nav_size_y+1) / nav_factor|, respectively

    Both |\mapbox| and |\navbox| also contain hyperlinks.

  {\it Count registers:}

  * |\pointerE|, |\pointerW|, |\pointerN|, |\pointerS| contain the page number 
    of the neighbouring pages in the E, W, N and S directions. If there is no 
    such a page its page number is set to 0.

  * |\pagenum| current page number

  {\it Token registers:}

  * |\pointerU|, |\pointerD| contain information about pages above and below
    the current page. It consists of one or more concatenated records. Each 
    record has a special format
    
    {\tt|page-name|\char124|page-number|\char124|destination|\char124\char124}
    
    If there are no such pages, the value is set to |notdef|.

    See the description of the |\processpointeritem| macro below for how to 
    extract and use this information.

  * |\pagename| = name of the current map according to options of the |map|
    command.

  * |\pagelabel| = the page label as specified by |origin| and |origin-label|
    options of the |layout| command.
\endlist


The following variables are set at the beginning of the document:

\list
  * |\hsize|, |\vsize| = \TeX\ page dimensions, set according to |page-width|
    and |page-height| parameters of the |page-setup| option of the |layout|
    command. They determine our playground when defining page layout using 
    the |\dopage| macro.

  * |\ifpagenumbering| = This conditional is set true or false according to
    the |page-numbers| option of the |layout| command.
\endlist

There are also some predefined macros which help with the processing of 
|\pointer*| variables:

\list
  * |\showpointer| with one of the |\pointerE|, |\pointerW|, |\pointerN| or 
    |\pointerS| as an argument displays the value of the argument. If the value
    is 0 it doesn't display anything. This is useful because
    the zero value (no neighbouring page) shouldn't be displayed.
    
  * |\showpointerlist| with one of the |\pointerU| or |\pointerD| as an argument 
    presents the content of this argument. (Which contains 
    |\pointerU| or |\pointerD|, see above.) For each record it calls the macro 
    |\processpointeritem|, which is responsible for data formatting. 
    
    Macro |\showpointerlist| should be used without redefinition in the place
    where you want to display the content of its argument; for custom data
    formatting redefine |\processpointeritem| macro.
         
  * |\processpointeritem| has three arguments (page-name, page-number, 
    destination) and visualizes these data. The arguments are delimited as 
    follows
    
    {\tt|\def\processpointeritem#1|\char124|#2|\char124|#3\endarg{...}|}

    An example definition may be

    {\tt|\def\processpointeritem#1|\char124|#2|\char124}|#3\endarg{%
  \hbox{\pdfstartlink attr {/Border [0 0 0]}% 
        goto name {#3} #2 (#1)\pdfendlink}%
}|

   (note how to use the {\it destination} argument), 
   or much simpler (if we don't need hyperlink features):

    {\tt|\def\processpointeritem#1|\char124|#2|\char124}|#3\endarg{%
  \hbox{#2 (#1)}%
}|
\endlist

For font management there are macros 

\list
* |\size[#1]| for size changes,

* |\color[#1 #2 #3]| for colour changes (RGB values in the range 0--100), and

* |\rm|, |\it|, |\bf|, |\ss|, |\si| for type face switching.
\endlist

%If you use own texts in \TeX\ macros (like a label for the scale bar), these 
%font switches work only in restricted manner: it's supposed that all characters 
%are present in the first encoding specified in the initialization file. 

See below for a list of predefined texts which may be used in the atlas.

There is also a |\framed| macro which takes a box as an argument and displays the 
box framed. The frame style can be customized by redefining the |\linestyle| 
macro which defaults to |1 J 1 j 1.5 w|.


Now we're ready to define the |\dopage| macro. You may choose which of the 
predefined elements to use. A very simple example would be

|layout my_layout
  scale 1 200
  page-setup 29.7 21 27.7 19 1 1 cm
  size 26.7 18 cm
  overlap 0.5 cm
  code tex-atlas
    \def\dopage{\box\mapbox}
    \insertmaps 
endlayout|

which defines the landscape A4 layout without the navigator nor any texts. There 
is only a map on the page.

Note the |\insertmaps| macro. Map pages are inserted at its position. 
This is not done automatically because you may wish to insert some other pages 
before the first map page. 

More advanced is the default definition of the |\dopage| macro:

|\def\dopage{%
 \vbox{\centerline{\framed{\mapbox}}
  \bigskip
  \line{%
    \vbox to \ht\navbox{
      \hbox{\size[20]\the\pagelabel
        \ifpagenumbering\space(\the\pagenum)\fi
        \space\size[16]\the\pagename}
      \ifpagenumbering
        \medskip
        \hbox{\qquad\qquad
          \vtop{%
            \hbox to 0pt{\hss\showpointer\pointerN\hss}
            \hbox to 0pt{\llap{\showpointer\pointerW\hskip0.7em}%
              \raise1pt\hbox to 0pt{\hss$\updownarrow$\hss}%
              \raise1pt\hbox to 0pt{\hss$\leftrightarrow$\hss}%
              \rlap{\hskip0.7em\showpointer\pointerE}}
              \hbox to 0pt{\hss\showpointer\pointerS\hss}
          }\qquad\qquad
          \vtop{
            \def\arr{$\uparrow$}
            \showpointerlist\pointerU
            \def\arr{$\downarrow$}
            \showpointerlist\pointerD
          }
        }
      \fi
      \vss
      \scalebar
    }\hss
    \box\navbox
  }
 }
}|

Using other plain \TeX\ macros or \TeX\ primitives it's possible to add other 
features, e.g.~a different layout for odd and even pages; headers and footers;
or adding a logo to each page. 

In addition to map pages contains atlas additional items: title page, basic 
facts about the cave, legend with used map symbols etc.

Therion automatically generates list of used map symbols and lists of persons
who have discovered, surveyed and drawn selected part of the cave. 
Following token registers may be used (according to user's requirements 
before or after the |\insertmaps| macro):

\list
* |\explotitle|, |\topotitle|, |\cartotitle| = translated titles
* |\exploteam|, |\topoteam|, |\cartoteam| = participating members 
  (according to |team|, |explo-team| options for |centreline| and |author|
  option of scraps)
* |\explodate|, |\topodate|, |\cartodate| = corresponding dates 
* |\comment| = is set according to |map-comment| option of the |layout|
  command
* |\copyrights| = is set according to copyright options for surveys and other
  objects
* |\cavename| = name of the exported map; set according to |-title| option
  of exported map
* |\cavelength|, |\cavedepth| = approximate length and depth of displayed map
* |\cavelengthtitle|, |\cavedepthtitle| = translated labels
* \NEW{5.4}|\cavemaxz|, |\caveminz| = altitude max/min value
* \NEW{5.4}|\thversion| = current therion version
* \NEW{5.4}|\currentdate| = current date
* \NEW{5.4}|\outcscode|, |\outcsname| = output coordinat system code and name
* \NEW{5.4}|\northdir| = `true' or `grid'
* \NEW{5.4}|\magdecl| = magnetic declination in degrees
* \NEW{5.4}|\gridconv| = grid meridian convergence in degrees
\endlist

There is a macro |\atlastitlepages| which combines most of the token registers 
mentioned above to get simple preformatted atlas introductory pages.

For legend displaying there are

\list
* |\iflegend| = conditional; true iff |legend| option of the |layout| command
  was set to |on| or |all| values
* |\legendtitle| = token register containing translated legend title
* |\insertlegend| = macro for inserting legend symbols pictures with translated
  descriptions in the specified number of columns (according to |legend-columns|
  layout option)
* |\formattedlegend| = combines all three above commands to get
  preformatted legend with header and symbols typeset in two\[Default; 
  adjust the |legend-columns| layout option to get them more or less]
  columns if |legend| option is set |on|
\endlist

North arrow and scale bar may be displayed using

\list
* |\ifnortharrow| = conditional; true if map projection is plan and
  symbol north-arrow is not hidden in |layout|
* |\ifscalebar| = conditional; true if scalebar is not hidden
* |\northarrow| = PDF form with the north arrow
* |\scalebar| = PDF form with the scale bar
\endlist

There is a general-purpose macro for typesetting in multiple columns\[Not to be 
used with map legend, where multiple columns are to be adjusted by 
|legend-columns| layout option]:
\list
* |\begmulti <i>|, |\endmulti| = text between these macros is typeset in
  |<i>| columns
\endlist

Example how to create atlas with lists of surveyors etc.\ followed by map pages 
and with legend at the end:

|code tex-atlas
  \atlastitlepages
  \insertmaps
  \formattedlegend|

\subchapter Page layout in the map mode.

In the map mode it's possible to use a lot of predefined variables which 
are described in the previous chapter: 

{\rightskip0cm plus 5cm
|\cavename|, |\comment|, |\copyrights|, 
|\explotitle|, |\topotitle|, |\cartotitle|, 
|\exploteam|, |\topoteam|, |\cartoteam|, 
|\explodate|, |\topodate|, |\cartodate|, 
|\cavelength|, |\cavedepth|, |\cavelengthtitle|, |\cavedepthtitle|,
|\cavemaxz|, |\caveminz|, |\thversion|, |\currentdate|,
|\outcscode|, |\outcsname|, |\northdir|, |\magdecl|, |\gridconv|,
|\ifnortharrow|, |\ifscalebar|, |\northarrow|, |\scalebar|, 
|\iflegend|, |\legendtitle|, |\insertlegend|, |\begmulti <i>|, |\endmulti|, 
|\formattedlegend|, |\legendcolumns|.
\par}

In order to place them somewhere on the map page, you have to define 
|\maplayout| macro in the |code tex-map| section of the |layout| command. 
It should contain one or more |\legendbox| invocations.
The |\legendbox| macro has four parameters: 
coordinates ranging 0--100, alignment specification 
(N, E, S, W, NE, SE, SW, NW or C) and the content to be displayed.

A simple example is

|\def\maplayout{
  \legendbox{0}{100}{NW}{\northarrow}
}|

which displays north arrow in the upper-left corner of the map sheet.

For user's convenience, there is |\legendcontent| token register. It contains 
preformatted cave name, north arrow, scale bar, explo/topo/carto teams, 
comment, copyrights and legend.
(The |\legendcontent| is also used in the default map layout definition:
|\def\maplayout{\legendbox{0}{100}{NW}{\the\legendcontent}}|).

Width of the above text may be adjusted by |\legendwidth| dimen register 
(its default value is set by |legend-width| layout option). 
The color and size of texts in the preformatted legend can be easily changed
using |\legendtextcolor|, |\legendtextsize|, |\legendtextsectionsize| and 
|\legendtextheadersize| token registers, 
e.g. for large blue text:

|code tex-map
  \legendwidth=20cm
  \legendtextcolor={\color[0 0 100]}
  \legendtextsize={\size[20]}
  \legendtextheadersize={\size[60]}|


It is possible to display the whole map framed by setting the |\framethickness| 
dimen register to positive value, e.g. |0.5mm|.


\subchapter Customizing text labels.

Starting with the release 5.4.1 you can use |fonts-setup| layout option
instead of the \MP\ macro |fonts_setup()|.

\subchapter New map symbols.

Therion's layout command makes it easy to switch among various predefined map 
symbol sets. If there is no such symbol or symbol set you want, it's possible 
to design new map symbols. 

However, this requires knowledge of the \MP\ language, which is used for map 
visualization. It's described in 

\list
  Hobby, J. D.: {\it A User's Manual for MetaPost}, available at \hfil\break
     \www{http://cm.bell-labs.com/cm/cs/cstr/162.ps.gz}
\endlist

User may also benefit from comprehensive reference to the {\manfnt METAFONT} 
language, which is quite similar to \MP:

\list
  Knuth, D. E.: {\it The METAFONTbook}, Reading, Massachusetts, Addison-Wesley
    $^1$1986
\endlist

New symbols may be defined in the |code metapost| section of the |layout| 
command. This makes it easy to add new symbols at the run-time. It is also 
possible to add symbols permanently by compiling them into Therion executable 
(see the {\it Appendix} for instructions how to do this).

Each symbol has to have a unique name, which consists of following items:

\list 
* one of the letters `p', `l', `a', `s' for point, line, area or special 
  symbols, respectively;
* underscore character;
* symbol type as listed in the chapter {\it Data format\/} with all dashes
  removed;
* if the symbol has a subtype, add underscore character and subtype;
* underscore character;
* symbol set identifier in uppercase
\endlist

Example: standard name for a point `water-flow' symbol with a `permanent' 
subtype in the `MY' set is |p_waterflow_permanent_MY|. Standard name for user-defined
symbol types should not include symbol set identifier, e.g.~|p_u_bat|.

Each new symbol has to be registered by a macro call

|initsymbol("<standard-name>");|

unless it's compiled into Therion executable. 

There are four predefined pens {\it PenA} (thickest) \dots\ {\it PenD} 
(thinnest), which should be used for all drawings. 
For drawing and filling use |thdraw| and |thfill| commands instead of \MP's 
|draw| and |fill|.

\NEW{5.4}The following variables are also available:

\list
* boolean |ATTR__shotflag_splay|, |ATTR__shotflag_duplicate|,\hfil\break
  |ATTR__shotflag_approx| = set for line survey
* boolean |ATTR__stationflag_splay| = set true for endstations of splay shots
* boolean |ATTR__scrap_centerline| = set true for scraps created from
  centreline
* boolean |ATTR__elevation| = true for (extended) elevation, false for
  plan projection
* numeric |ATTR__height| = height of a pit or wall:pit
* string |ATTR__id| = contains current object ID
* string |ATTR__survey| = contains current survey name
* string |ATTR__scrap| = contains current scrap name
* picture |ATTR__text| = contains typeset text e.g.\ for point continuation
* string |NorthDir| = `true' or `grid'
* numeric |MagDecl| = magnetic declination in degrees
* numeric |GridConv| = grid meridian convergence in degrees
\endlist

\subsubchapter Point symbols.

Point symbols are defined as macros using |def ... enddef;| commands.
Majority of point symbol definitions has four arguments: 
position (pair), rotation (numeric), scale (numeric) and alignment (pair). 
Exceptions are {\it section} which has no visual representation; 
all {\it labels}, which require special treatment as described in the
previous chapter, and 
{\it station} which takes only one argument: position (pair).

All point symbols are drawn in local coordinates with the length unit {\it u}. 
Recommended ranges are $\left<-0.5u,0.5u\right>$ in both axes. The symbol 
should be centered at the coordinates' origin.
For the final map, all drawings are transformed as specified in the {\it T\/} 
transformation variable, so it's necessary to set this variable before drawing. 

This is usually done in two steps (assume that four arguments are 
{\it P}, {\it R}, {\it S}, {\it A}): 
  
\list
* set the {\it U} pair variable to $\left({width\over2},{height\over2}\right)$ 
  of the symbol for correct alignment. The alignment argument {\it A} is a pair 
  representing ratios $\left(shift_x\over U_x\right)$ and
  $\left(shift_y\over U_y\right)$. 
  
  (Hence |aligned A| means |shifted (xpart A * xpart U, ypart A * ypart U)|.)
* set the {\it T} transformation variable 

  {\catcode`\=12|T:=identity aligned A rotated R scaled S shifted P;|}
\endlist

For drawing and filling use |thdraw| and |thfill| commands instead of \MP's 
|draw| and |fill|. These take automatically care of {\it T} transformation.

An example definition may be

|def p_entrance_UIS (expr P,R,S,A)=
  U:=(.2u,.5u);
  T:=identity aligned A rotated R scaled S shifted P;
  thfill (-.2u,-.5u)--(0,.5u)--(.2u,-.5u)--cycle;
enddef;
initsymbol("p_entrance_UIS");|

\subsubchapter Line symbols.

Line symbols differ from point symbols in respect that there is no local 
coordinate system. Each line symbol gets the {\it path} in absolute coordinates 
as the first argument. Therefore it's necessary to set {\it T} variable to 
|identity| before drawing.

Following symbols take additional arguments: 
\list
* arrow = numeric: 0 is no arrows, 1 arrow at the end, 2 begin, 3 both ends
* contour = text: list of points which get the tick or one of 
  $-1$, $-2$ or $-3$ to mark undefined tick, tick in the middle or 
  no tick, respectively
* section = text: list of points which get the orientation arrow or 
  $-1$ to indicate no arrows
* slope = numeric: 0 no border, 1 border; text: list of (point,direction,length) 
  triplets
\endlist

Usage example:

|def l_wall_bedrock_UIS (expr P) = 
  T:=identity;
  pickup PenA;
  thdraw P;
enddef;
initsymbol("l_wall_bedrock_UIS");|

\subsubchapter Area symbols.

Areas are similar to lines: they take only one argument -- {\it path} in 
absolute coordinates. 

You may fill them in three ways:

\list
 * fill an uniform or randomised grid in a temporary picture 
   (having dimensions |bbox path|) with some point symbols; clip it according 
   to |path| and add to the |currentpicture|
 * fill |path| with a solid colour
 * fill |path| with a predefined pattern using a |withpattern| keyword.
\endlist

Patterns are defined using the same user interface (without the 
|patterncolor| macro) as described in the article

\list
  Bolek, P.: ``\MP\ and patterns,'' {\it TUGboat}, 3, XIX (1998), pp.~276--283,
  available online at
  \www{https://www.tug.org/TUGboat/Articles/tb19-3/tb60bolek.pdf}
\endlist

You may use standard \MP\ |draw| and similar macros without setting of {\it T} 
variable in pattern definitions.

Example on how to define and use patterns:

|beginpattern(pattern_water_UIS);
  draw origin--10up withpen pensquare scaled (0.02u);
  patternxstep(.18u);
  patterntransform(identity rotated 45);
endpattern;

def a_water_UIS (expr p) =
  T:=identity;
  thclean p;
  thfill p withpattern pattern_water_UIS;
enddef;
initsymbol("a_water_UIS");|

\subsubchapter Special symbols.

There are currently two special symbols: scale bar and north arrow. 
Both are experimental and subject to change.

\endinput
